\documentclass[a4paper, titlepage]{article}

\usepackage{graphicx} % Required to insert images
\usepackage{natbib}
\usepackage{listings}

\usepackage{amsmath}
\usepackage{tabularx}
\usepackage[utf8]{inputenc}

\begin{document}

\title{Week 41}
\author{Sigurt Bladt Dinesen \\sidi{@}itu.dk \and Jens Egholm Pedersen \\jegp{@}itu.dk}
\maketitle
\section*{8.1}
\subsection*{ex3.c}
\begin{lstlisting}[numbers=left, firstnumber=0, title=ex3.c]
            LDARGS
            CALL (1,"L1")         //call main
            STOP                  //halt after main returns

Label "L1"  INCSP 1               //int i
            GETBP                 //i=0
            CSTI 1
            ADD
            CSTI 0
            STI
            INCSP -1              //i=0 done, and stack 'cleaned'
            GOTO "L3"

Label "L2"  GETBP
            CSTI 1
            ADD
            LDI
            PRINTI                //print i;
            INCSP -1
            GETBP
            CSTI 1
            ADD
            GETBP
            CSTI 1
            ADD
            LDI
            CSTI 1
            ADD                   //i+1
            STI                   //i=i+1
            INCSP -1
            INCSP 0

Label "L3"  GETBP
            CSTI 1
            ADD
            LDI                   //i
            GETBP
            CSTI 0
            ADD
            LDI                   //n
            LT                    //i<n
	    IFNZRO "L2"           //i<n done (jump if..)
            INCSP -1
            RET 0                 //void-return from main
\end{lstlisting}

\subsection*{ex.5.c}
\begin{lstlisting}[numbers=left, firstnumber=0, title=ex5.c]
            LDARGS
            CALL (1,"L1")         //call main
            STOP                  //halt after main returns here

Label "L1"  INCSP 1               //int r
            GETBP                 //r = n starts
            CSTI 1
            ADD
            GETBP
            CSTI 0
            ADD
            LDI
            STI                   //r = n done
            INCSP -1
            INCSP 1               //int r inside block
            GETBP
            CSTI 0
            ADD
            LDI                   //n
            GETBP
            CSTI 2                //&r
            ADD
            CALL (2,"L2")         //square(n, &r)
            INCSP -1
            GETBP
            CSTI 2
            ADD
            LDI                   //r
            PRINTI                //print r - i.e. the square of n
            INCSP -1
            INCSP -1
            GETBP
            CSTI 1
            ADD
            LDI                   //r outside block
            PRINTI                //print r - i.e. n
            INCSP -1
            INCSP -1
            RET 0                 //void-return from main

Label "L2"  GETBP
            CSTI 1
            ADD
            LDI                   //*rp (as a reference)
            GETBP
            CSTI 0
            ADD
            LDI                   //i
            GETBP
            CSTI 0
            ADD
            LDI                   //i
            MUL                   //i * i
            STI                   //*rp = i * i
            INCSP -1
            INCSP 0
            RET 1                 //void-return from square
\end{lstlisting}
\subsection*{ex.3.c trace}
%[stack] {pc: instr param}
\begin{lstlisting}[numbers=left, title=ex.3.c.trace]
[ ]{0: LDARGS}
[ 4 ]{1: CALL 1 5}                     //call main with 1 arg (4)
[ 4 -999 4 ]{5: INCSP 1}               //make space for variable i
[ 4 -999 4 0 ]{7: GETBP}
[ 4 -999 4 0 2 ]{8: CSTI 1}            //addr for i is bp+1 (bp holds n)
[ 4 -999 4 0 2 1 ]{10: ADD}
[ 4 -999 4 0 3 ]{11: CSTI 0}
[ 4 -999 4 0 3 0 ]{13: STI}            //i=0
[ 4 -999 4 0 0 ]{14: INCSP -1}         //pop 0 off
[ 4 -999 4 0 ]{16: GOTO 43}            //jump to while-guard
[ 4 -999 4 0 ]{43: GETBP}
[ 4 -999 4 0 2 ]{44: CSTI 1}
[ 4 -999 4 0 2 1 ]{46: ADD}
[ 4 -999 4 0 3 ]{47: LDI}              //load i onto stack
[ 4 -999 4 0 0 ]{48: GETBP}
[ 4 -999 4 0 0 2 ]{49: CSTI 0}
[ 4 -999 4 0 0 2 0 ]{51: ADD}
[ 4 -999 4 0 0 2 ]{52: LDI}            //load n onto stack
[ 4 -999 4 0 0 4 ]{53: LT}
[ 4 -999 4 0 1 ]{54: IFNZRO 18}        //jump to while-body if i<n
[ 4 -999 4 0 ]{18: GETBP}
[ 4 -999 4 0 2 ]{19: CSTI 1}
[ 4 -999 4 0 2 1 ]{21: ADD}
[ 4 -999 4 0 3 ]{22: LDI}              //load i onto stack
[ 4 -999 4 0 0 ]{23: PRINTI}
0 [ 4 -999 4 0 0 ]{24: INCSP -1}       //pop off i after printing it
[ 4 -999 4 0 ]{26: GETBP}
[ 4 -999 4 0 2 ]{27: CSTI 1}
[ 4 -999 4 0 2 1 ]{29: ADD}            //leaves address of i on stack
[ 4 -999 4 0 3 ]{30: GETBP}
[ 4 -999 4 0 3 2 ]{31: CSTI 1}
[ 4 -999 4 0 3 2 1 ]{33: ADD}          //same
[ 4 -999 4 0 3 3 ]{34: LDI}            //replace addr(i) with i (s[3]=0)
[ 4 -999 4 0 3 0 ]{35: CSTI 1}
[ 4 -999 4 0 3 0 1 ]{37: ADD}          //leaves i+1 on stack (0+1=1)
[ 4 -999 4 0 3 1 ]{38: STI}            //store 1 at s[3] (i=1)
[ 4 -999 4 1 1 ]{39: INCSP -1}         //pop 1 off
[ 4 -999 4 1 ]{41: INCSP 0}            //that's dumb
[ 4 -999 4 1 ]{43: GETBP}              //oh look, it's the while-guard again!
[ 4 -999 4 1 2 ]{44: CSTI 1}
[ 4 -999 4 1 2 1 ]{46: ADD}
[ 4 -999 4 1 3 ]{47: LDI}              //still fetches i
[ 4 -999 4 1 1 ]{48: GETBP}
[ 4 -999 4 1 1 2 ]{49: CSTI 0}
[ 4 -999 4 1 1 2 0 ]{51: ADD}
[ 4 -999 4 1 1 2 ]{52: LDI}            //and n
[ 4 -999 4 1 1 4 ]{53: LT}
[ 4 -999 4 1 1 ]{54: IFNZRO 18}        //and i<n (1<4) so go while-body!
[ 4 -999 4 1 ]{18: GETBP}
[ 4 -999 4 1 2 ]{19: CSTI 1}
[ 4 -999 4 1 2 1 ]{21: ADD}
[ 4 -999 4 1 3 ]{22: LDI}
[ 4 -999 4 1 1 ]{23: PRINTI}
1 [ 4 -999 4 1 1 ]{24: INCSP -1}
[ 4 -999 4 1 ]{26: GETBP}
[ 4 -999 4 1 2 ]{27: CSTI 1}
[ 4 -999 4 1 2 1 ]{29: ADD}
[ 4 -999 4 1 3 ]{30: GETBP}
[ 4 -999 4 1 3 2 ]{31: CSTI 1}
[ 4 -999 4 1 3 2 1 ]{33: ADD}
[ 4 -999 4 1 3 3 ]{34: LDI}
[ 4 -999 4 1 3 1 ]{35: CSTI 1}
[ 4 -999 4 1 3 1 1 ]{37: ADD}
[ 4 -999 4 1 3 2 ]{38: STI}
[ 4 -999 4 2 2 ]{39: INCSP -1}
[ 4 -999 4 2 ]{41: INCSP 0}
[ 4 -999 4 2 ]{43: GETBP}              //rinse - repeat
[ 4 -999 4 2 2 ]{44: CSTI 1}
[ 4 -999 4 2 2 1 ]{46: ADD}
[ 4 -999 4 2 3 ]{47: LDI}
[ 4 -999 4 2 2 ]{48: GETBP}
[ 4 -999 4 2 2 2 ]{49: CSTI 0}
[ 4 -999 4 2 2 2 0 ]{51: ADD}
[ 4 -999 4 2 2 2 ]{52: LDI}
[ 4 -999 4 2 2 4 ]{53: LT}
[ 4 -999 4 2 1 ]{54: IFNZRO 18}
[ 4 -999 4 2 ]{18: GETBP}
[ 4 -999 4 2 2 ]{19: CSTI 1}
[ 4 -999 4 2 2 1 ]{21: ADD}
[ 4 -999 4 2 3 ]{22: LDI}
[ 4 -999 4 2 2 ]{23: PRINTI}
2 [ 4 -999 4 2 2 ]{24: INCSP -1}
[ 4 -999 4 2 ]{26: GETBP}
[ 4 -999 4 2 2 ]{27: CSTI 1}
[ 4 -999 4 2 2 1 ]{29: ADD}
[ 4 -999 4 2 3 ]{30: GETBP}
[ 4 -999 4 2 3 2 ]{31: CSTI 1}
[ 4 -999 4 2 3 2 1 ]{33: ADD}
[ 4 -999 4 2 3 3 ]{34: LDI}
[ 4 -999 4 2 3 2 ]{35: CSTI 1}
[ 4 -999 4 2 3 2 1 ]{37: ADD}
[ 4 -999 4 2 3 3 ]{38: STI}
[ 4 -999 4 3 3 ]{39: INCSP -1}
[ 4 -999 4 3 ]{41: INCSP 0}
[ 4 -999 4 3 ]{43: GETBP}              //rinse - repeat
[ 4 -999 4 3 2 ]{44: CSTI 1}
[ 4 -999 4 3 2 1 ]{46: ADD}
[ 4 -999 4 3 3 ]{47: LDI}
[ 4 -999 4 3 3 ]{48: GETBP}
[ 4 -999 4 3 3 2 ]{49: CSTI 0}
[ 4 -999 4 3 3 2 0 ]{51: ADD}
[ 4 -999 4 3 3 2 ]{52: LDI}
[ 4 -999 4 3 3 4 ]{53: LT}
[ 4 -999 4 3 1 ]{54: IFNZRO 18}
[ 4 -999 4 3 ]{18: GETBP}
[ 4 -999 4 3 2 ]{19: CSTI 1}
[ 4 -999 4 3 2 1 ]{21: ADD}
[ 4 -999 4 3 3 ]{22: LDI}
[ 4 -999 4 3 3 ]{23: PRINTI}
3 [ 4 -999 4 3 3 ]{24: INCSP -1}
[ 4 -999 4 3 ]{26: GETBP}
[ 4 -999 4 3 2 ]{27: CSTI 1}
[ 4 -999 4 3 2 1 ]{29: ADD}
[ 4 -999 4 3 3 ]{30: GETBP}
[ 4 -999 4 3 3 2 ]{31: CSTI 1}
[ 4 -999 4 3 3 2 1 ]{33: ADD}
[ 4 -999 4 3 3 3 ]{34: LDI}
[ 4 -999 4 3 3 3 ]{35: CSTI 1}
[ 4 -999 4 3 3 3 1 ]{37: ADD}
[ 4 -999 4 3 3 4 ]{38: STI}
[ 4 -999 4 4 4 ]{39: INCSP -1}
[ 4 -999 4 4 ]{41: INCSP 0}
[ 4 -999 4 4 ]{43: GETBP}              //while-guard again
[ 4 -999 4 4 2 ]{44: CSTI 1}
[ 4 -999 4 4 2 1 ]{46: ADD}
[ 4 -999 4 4 3 ]{47: LDI}
[ 4 -999 4 4 4 ]{48: GETBP}
[ 4 -999 4 4 4 2 ]{49: CSTI 0}
[ 4 -999 4 4 4 2 0 ]{51: ADD}
[ 4 -999 4 4 4 2 ]{52: LDI}
[ 4 -999 4 4 4 4 ]{53: LT}             //this time we don't jump though
[ 4 -999 4 4 0 ]{54: IFNZRO 18}        //as 4>=4
[ 4 -999 4 4 ]{56: INCSP -1}           //pop i
[ 4 -999 4 ]{58: RET 0}                //void-return from main
[ 4 ]{4: STOP}                         //halt
\end{lstlisting}

\section*{8.4}
\subsection*{ex8.c vs prog1}
\begin{lstlisting}[numbers=left, title=ex8.c]
            LDARGS
            CALL (0,"L1")
            STOP

Label "L1"  INCSP 1
            GETBP
            CSTI 0
            ADD
            CSTI 20000000
            STI
            INCSP -1
            GOTO "L3"

Label "L2"  GETBP
            CSTI 0
            ADD
            GETBP
            CSTI 0
            ADD
            LDI
            CSTI 1
            SUB
            STI
            INCSP -1
            INCSP 0

Label "L3"  GETBP
            CSTI 0
            ADD
            LDI
            IFNZRO "L2"
            INCSP -1
            RET -1
\end{lstlisting}
\begin{lstlisting}[numbers=left, title=prog1]
0 20000000         //CSTI 20000000
16 7               //GOTO 7
0 1                //CSTI 1
2                  //ADD
9                  //DUP
18 4               //IFNZERO 4
25                 //STOP
\end{lstlisting}
The generated instructions include code to call, and return
form, a function main. More importantly: it implements the
while loop by continuously incrementing and decrementing the stack-pointer and
using LDI to load the contents of $i$, in both the while guard and body,
whereas the hand-typed machine instructions in prog1 completely ignores the
concepts of pointers (and functions).
\begin{lstlisting}[numbers=left, title=ex13.c]
            LDARGS
            CALL (1,"L1")
            STOP

Label "L1"  INCSP 1               //var initiation
            GETBP
            CSTI 1
            ADD
            CSTI 1889
            STI
            INCSP -1
            GOTO "L3"

Label "L2"  GETBP
            CSTI 1
            ADD
            GETBP
            CSTI 1
            ADD
            LDI
            CSTI 1
            ADD
            STI                   //y=y+1
            INCSP -1
            GETBP
            CSTI 1
            ADD
            LDI
            CSTI 4
            MOD
            CSTI 0
            EQ
            IFZERO "L7"
            GETBP
            CSTI 1
            ADD
            LDI
            CSTI 100
            MOD
            CSTI 0
            EQ
            NOT
            IFNZRO "L9"
            GETBP
            CSTI 1
            ADD
            LDI
            CSTI 400
            MOD
            CSTI 0
            EQ
            GOTO "L8"

Label "L9"  CSTI 1

Label "L8"  GOTO "L6"

Label "L7"  CSTI 0

Label "L6"  IFZERO "L4"
            GETBP
            CSTI 1
            ADD
            LDI
            PRINTI
            INCSP -1
            GOTO "L5"

Label "L4"  INCSP 0

Label "L5"  INCSP 0

Label "L3"  GETBP
            CSTI 1
            ADD
            LDI
            GETBP
            CSTI 0
            ADD
            LDI
            LT
            IFNZRO "L2"
            INCSP -1
            RET 0
\end{lstlisting}
In addition to loading up $y$ every time it is used, rather than just leaving
it on the stack, the code does an insane amount of jumping to implement the
short-circuiting boolean operations. It should be adequate to have \emph{one}
label for (re)entering the while loop, and one for skipping it (when $y\ge n$).

\section*{8.6}
We have not been able to make this (yet), though it is probably more of an
"wat Iz dis f\#?" problem than a stack-machine problem.

\section*{Questions}
Why can't (shouln't) the a ? b : c expression be a keyword in the lexer?
\end{document}
