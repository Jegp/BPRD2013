\documentclass{article}

\usepackage{graphicx} % Required to insert images
\usepackage{courier} % Required for the courier font

\begin{document}

\section*{Re-hand-in}
This version is an attempt to correct our grievous mistakes in 3.2 where we updated our NFA to not allow the empty string, as per our regex. We also have a question for you in this section. 
In 3.3 we used the rules defined in section 3.6 in the book and restructured the derivations to fit the actual AST.
In 3.4 the tree have been fixed. And lastly an empty environment was added to the compString method definition.

Happy reading :-)

\section*{3.2}

Regex: $a|(ba|b)+$
\newline\textit{to allow the empty string, replace $+$ with $*$}

\vspace{10 mm}

\begin{tabular}{c c}
\textbf{NFA} & \textbf{DFA} \\
\includegraphics[height=250px]{3-2-nfa.png} &
	\includegraphics[height=250px]{3-2-dfa.png}
\end{tabular}
\subsection*{Question for TA}
We are in doubt about some of the conventions when drawing automata:
If our NFA is run on the string "aa", we would expect either failure, or
two separate matches, one for each "a", depending on the way the NFA is applied.
Here is the question: When in state \emph{a}, if the next input is "a", does the
NFA 'start over' or is just left in that state, effectively accepting the string
"aa" bacause \emph{a} has an epsilon transition to an accepting state?

\section*{3.3}
\begin{tabular}{l l l l}
\textbf{Rule} & \textbf{Label} & \textbf{Input} & \textbf{Derivation} \\
A & EXPR\#0 & LET z = (17) IN z + 2 * 3 END EOF & EXPR\#1 EOF \\
F & EXPR\#1 & LET z = (17) IN z + 2 * 3 END & LET NAME = EXPR\#2 IN EXPR\#3 \\
G & EXPR\#4 & z + 2 * 3 & Prim("*", EXPR\#5, EXPR\#6) \\
C & EXPR\#5 & z + 2 * 3 & CstI 2 \\
C & EXPR\#6 & z + 2 * 3 & CstI 3 \\
H & EXPR\#3 & z + & Prim("+", EXPR\#7, \\
  &         &     & \quad Prim("*", CstI 2, CstI 3)) \\
B & EXPR\#7 & z & Var z \\
E & EXPR\#2 & (17) & 17 \\
C & EXPR\#2 & 17 & CstI 17 \\
B & EXPR\#1 & z & Var z 
\end{tabular}

\section*{3.4}
\begin{verbatim}
let z = 17 in z + 2 * 3 end EOF
\end{verbatim}
\includegraphics[height=300px]{3-4.png}

\section*{3.6}

\begin{verbatim}
let compString code = 
  scomp (fromString code) []
\end{verbatim}

\end{document}
