\documentclass{article}

\usepackage{graphicx} % Required to insert images
\usepackage{courier} % Required for the courier font

\begin{document}

\section*{3.2}

Regex: $a|(ba|b)+$
\newline\textit{to allow the empty string, replace $+$ with $*$}

\vspace{10 mm}

\begin{tabular}{c c}
\textbf{NFA} & \textbf{DFA} \\
\includegraphics[height=250px]{3-2-nfa.png} &
	\includegraphics[height=250px]{3-2-dfa.png}
\end{tabular}
\subsection*{Question for TA}
We are in doubt about some of the conventions when drawing automata:
If our NFA is run on the string "aa", we would expect either failure, or
two separate matches, one for each "a", depending on the way the NFA is applied.
Here is the question: When in state \emph{a}, if the next input is "a", does the
NFA 'start over' or is just left in that state, effectively accepting the string
"aa" bacause \emph{a} has an epsilon transition to an accepting state?

\section*{3.3}

\begin{tabular}{l l l}
\textbf{Label} & \textbf{Input} & \textbf{Derivation} \\
EXPR\#0 & LET z = (17) IN z + 2 * 3 END EOF & MAIN EXPR\#1 EOF \\
EXPR\#1 & LET z = (17) IN z + 2 * 3 END & LET NAME = EXPR\#2 IN EXPR\#3 \\
EXPR\#3 & z + 2 * 3 & VAR z + EXPR\#4 \\
EXPR\#4 & 2 * 3 & CstI 2 TIMES CstI 3 \\
EXPR\#3 & z + & Var z ADD EXPR\#4 \\
EXPR\#2 & (17) IN & CstI 17 \\
NAME    & z & VAR z \\
-       & - & MAIN

\end{tabular}

\section*{3.4}
\begin{verbatim}
let z = 17 in z + 2 * 3 end EOF
\end{verbatim}
\includegraphics[height=300px]{3-4.png}

\section*{3.6}

\begin{verbatim}
let compString code =
  scomp (fromString code)
\end{verbatim}

\end{document}
