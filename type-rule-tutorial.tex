\documentclass[a4paper]{article}

\usepackage{natbib}
\usepackage{graphicx}
\usepackage{rotating}

\usepackage{amsmath}
\usepackage{tabularx}
\usepackage[utf8]{inputenc}

\usepackage[ligature,inference,reserved,shorthand]{semantic}

\usepackage{color}
\usepackage[usenames,dvipsnames,svgnames,table]{xcolor}


% Grammars
\def\grammar#1#2{%
  \vspace{1em}
  \begin{tabular}{r@{$\quad$}r@{$\quad$}l@{$\qquad$}l}
  $#1$ & $::=$ & & \emph{#2}\\
}
\def\endgrammar{\end{tabular}\vspace{1em}}
\def\prod#1#2{&$\mid$&$#1$&\emph{#2}}

% Meta
\newcommand{\RULE}[1]{\textsc{#1}}
\newcommand{\SUBST}[2]{\left[#1\right]#2}

% Syntax styles
\colorlet{term}{red}
\newcommand{\kwd}[1]{\ensuremath{\mathsf{\textcolor{term}{\underline{#1}}}}}
\newcommand{\expr}[1]{\ensuremath{\textcolor{term}{#1}}}
\newcommand{\type}[1]{\ensuremath{\textcolor{blue}{#1}}}
\newcommand{\context}[1]{\ensuremath{\textcolor{black}{#1}}}
\newcommand{\scheme}[1]{\ensuremath{\textcolor{ForestGreen}{#1}}}
\newcommand{\black}[1]{\textcolor{black}{#1}}

\newcommand{\const}[1]{\mathsf{#1}}

\newcommand{\typing}[3]{\ensuremath{#1\ \vdash\ #2\ :\ #3}}


% Object language syntax
\newcommand{\TRUE}{\ensuremath{\expr{\const{true}}}}
\newcommand{\FALSE}{\ensuremath{\expr{\const{false}}}}
\newcommand{\PLUS}[2]{#1\;\expr{+}\;#2}
\newcommand{\LT}[2]{#1\;\expr{<}\;#2}
\newcommand{\NUM}[1]{\expr{\const{#1}}}

\newcommand\var[1]{\ensuremath{\expr{\mathsf{#1}}}}
\newcommand{\app}[2]{#1\ #2}

\newcommand{\IF}[3]{\kwd{if}\ #1\ \kwd{then}\ #2\ \kwd{else}\ #3}

\newcommand{\LET}[3]{\kwd{let}\ #1\ \textcolor{term}{=}\ #2\ \kwd{in}\ #3\ \kwd{end}}
\newcommand{\LETF}[4]{\kwd{let}\ #1\ #2\ \textcolor{term}{=}\ #3\ \kwd{in}\ #4\ \kwd{end}}

% Object language types
\newcommand\INT{\type{\const{int}}}
\newcommand\BOOL{\type{\const{bool}}}
\newcommand{\fn}[2]{#1\;\type{\to}\;#2}

% Type schemes
\newcommand\FORALL[2]{\scheme{\forall #1\;.\;#2}}
\newcommand\ALPHAS[1][n]{\alpha_1,\ldots,\alpha_{#1}}

%Contexts
\newcommand{\EMPTY}{\context{\cdot}}
\newcommand{\WITH}[2]{\context{#1\left[#2\right]}}
\newcommand{\RHO}{\context{\rho}}
\newcommand{\LOOKUP}[2]{#1\left(#2\right)}


\title{A Tutorial on Polymorphic Type Derivations}
\author{David Christiansen}
\date{DRAFT of December 27, 2012}
\begin{document}


\newcolumntype{L}{>{\raggedright\arraybackslash}X}


\maketitle

This is a tutorial on how to produce typing derivation trees for a simplified
version of ML's type system. Rather than being incredibly precise about every
detail, this tutorial will focus on building up an understanding of why the
rules work as they do. This tutorial is based on the presentation of
polymorphic typing in \emph{Programming Language Concepts}
\citep{Sestoft2012}. For your convenience, the grammar and type system of the
small version of ML used in this tutorial are reproduced in Appendix
\ref{sec:formal-rules}.

\paragraph{DRAFT}
This version of the tutorial has been used for one semester, but may still
contain minor errors or incomplete explanations. Please send any comments
about the style in which it is written or any corrections to
\texttt{drc@itu.dk}.

\paragraph{Notational Conventions}
When looking at a formal definition of a type system, it can be difficult at
first to distinguish the different categories of symbols from each
other. Rather than sweep this issue under the rug, colors and other
typographical conventions are used to precisely indicate which category a
symbol is a part of. The following table describes the applied conventions:
\noindent
\begin{tabularx}{\textwidth}{|L|l|L|}
  \hline
  \textbf{Category} & \textbf{Typeface} & \textbf{Examples}\\
  \hline\hline
  ML keywords & Red, underlined & \kwd{if}, \kwd{let}\\\hline
  ML expressions & Red, sans-serif, upright &
  $\PLUS{\NUM{17}}{\NUM{25}}$,
  $\IF{\LT{\NUM{23}}{\NUM{42}}}{\NUM{13}}{\NUM{99}}$\\\hline
  Variables representing expressions in rules& Italic roman red & $\expr{e_1}$, $\expr{e_2}$, $\expr{e_3}$ \\
  \hline
  ML Types & Blue, sans-serif, upright & $\INT$, $\BOOL$, $\fn{\type{\alpha}}{\INT}$\\
  \hline
  Variables representing ML types in rules & Italic roman blue& $\type{t_1}$\\
  \hline
  Type schemes and quantified type variables & Green &
  $\FORALL{\alpha}{\fn{\scheme{\alpha}}{\scheme{\alpha}}}$,
  $\FORALL{\alpha}{\fn{\scheme{\alpha}}{\BOOL}}$,
  $\FORALL{\alpha,\beta}{\fn{\fn{\scheme{\alpha}}{\scheme{\beta}}}{\scheme{\alpha}}}$\\
  \hline
\end{tabularx}

\noindent
The distinction between \emph{variables representing types or expressions} and
\emph{actual types or expressions}, which is represented by writing the
variables in italic type, is important.  A final proof will not contain any of
the variables, as they are used to demonstrate the structure of the proof.

\paragraph{What are we doing here, anyway?}
It is important to have the right perspective when looking at typing rules and how to combine them.
Typing rules serve two purposes: they serve as a notation for helping humans keep track of complicated ideas, and they aid communication of these complicated ideas between humans.
Even though they are quite formal and their use is highly precise, they are not primarily intended for a computer.
When we build up a tree structure of typing rules, we construct something called a \emph{typing derivation}.
In ML, a typing derivation is a proof that some expression \emph{can have} some type, given a particular context, not that the expression can \emph{only} have that type or that the expression will have that type no matter where we encounter it.
More formally, a derivation for $\typing{\RHO}{\expr{e}}{\type{t}}$ is a proof that in some context $\RHO$ (which assigns types to variables in $\expr{e}$), we can show that $\expr{e}$ can have (at least) the type $\type{t}$, and possibly others.
This proof is something that we, as creative, thinking beings, need to come up with.
We cannot necessarily construct it by mechanically following rules.
In fact, the discovery of an efficient algorithm for type-checking in some type system is a discovery that is at least as important as that of the type system, because it enables it to be actually used, and these algorithms can easily be non-obvious.
For an example of such an algorithm, see Section 6.4 of \citet{Sestoft2012}.
So, to recap, we are constructing a proof, for other humans, that a particular expression can have a particular type in ML.


\section{Examples without let}
\label{sec:examples-without-let}

We begin by using some simple examples to demonstrate how typing rules fit
together, without using more advanced features of the type system.

If we want to show that $\IF{\LT{\NUM{2}}{\NUM{3}}}{\NUM{42}}{\NUM{-5}}$ can have type $\INT$ in the empty typing context (called ``$\EMPTY$'', which is just a dot), we begin by creating a tree with this fact at the bottom.

\begin{displaymath}
  \inference{?}{
    \typing{\EMPTY}{
      \IF{\LT{\NUM{2}}{\NUM{3}}}{\NUM{42}}{\NUM{-5}}
    }{
      \INT
    }
  }
\end{displaymath}

Inspecting our typing rules, we see that only one of them has an \kwd{if}-expression at the bottom: rule \RULE{p7}. Applying this rule gives us three new typings to prove:
\begin{itemize}
\item $\typing{\EMPTY}{\LT{\NUM{2}}{\NUM{3}}}{\BOOL}$
\item $\typing{\EMPTY}{\NUM{42}}{\INT}$
\item $\typing{\EMPTY}{\NUM{-5}}{\INT}$
\end{itemize}
Note that the variable $\type{t}$ in \RULE{p7} has been filled out with
\INT. We do this because \INT\ is the only type can possibly be used, because the same $\type{t}$ occurs on the right-hand-side of the colon in the bottom of \RULE{p7}. Our derivation now looks like:
\begin{displaymath}
  \inference[\RULE{p7}]{
    \typing{\EMPTY}{\LT{\NUM{2}}{\NUM{3}}}{\BOOL} &
    \typing{\EMPTY}{\NUM{42}}{\INT} &
    \typing{\EMPTY}{\NUM{-5}}{\INT}
  }{
    \typing{\EMPTY}{
      \IF{\LT{\NUM{2}}{\NUM{3}}}{\NUM{42}}{\NUM{-5}}
    }{
      \INT
    }
  }
\end{displaymath}
The two branches of the $\kwd{if}$-expression are easy to show -- as they are
integer literals, we merely need to apply rule \RULE{p1} to them. Rule
\RULE{p5} is the only one that has a less-than expression on the bottom, so we know that we will use it for the condition of our \kwd{if}-expression. We now have the partial derivation:
\begin{displaymath}
  \inference[\RULE{p7}]{
    \inference[\RULE{p5}]{?}{\typing{\EMPTY}{\LT{\NUM{2}}{\NUM{3}}}{\BOOL}} &
    \inference[\RULE{p1}]{}{\typing{\EMPTY}{\NUM{42}}{\INT}} &
    \inference[\RULE{p1}]{}{\typing{\EMPTY}{\NUM{-5}}{\INT}}
  }{
    \typing{\EMPTY}{
      \IF{\LT{\NUM{2}}{\NUM{3}}}{\NUM{42}}{\NUM{-5}}
    }{
      \INT
    }
  }
\end{displaymath}
We now need only to show that the arguments to $\LT{}{}$ can both be integers
in the empty environment. We do this by using rule \RULE{p1} twice, completing
the derivation:\\
\begin{displaymath}
\resizebox{1.0\textwidth}{!}{
  \inference[\RULE{p7}]{
    \inference[\RULE{p5}]{
      \inference[\RULE{p1}]{}{
        \typing{\EMPTY}{\NUM{2}}{\INT}
      } &
      \inference[\RULE{p1}]{}{
        \typing{\EMPTY}{\NUM{3}}{\INT}
      }
    }{\typing{\EMPTY}{\LT{\NUM{2}}{\NUM{3}}}{\BOOL}} &
    \inference[\RULE{p1}]{}{\typing{\EMPTY}{\NUM{42}}{\INT}} &
    \inference[\RULE{p1}]{}{\typing{\EMPTY}{\NUM{-5}}{\INT}}
  }{
    \typing{\EMPTY}{
      \IF{\LT{\NUM{2}}{\NUM{3}}}{\NUM{42}}{\NUM{-5}}
    }{
      \INT
    }
  }
}
\end{displaymath}

On the other hand, we cannot argue that \PLUS{\TRUE}{\NUM{7}} has any
type. Let's walk through the process to see where it goes wrong. We begin with
a tree that has our desired term on the bottom. For now, we'll represent the
result type with a variable $\type{t}$ that we'll attempt to fill in later. If
we can't fill it in, then our attempt to find a type for the expression
failed. We'll use some arbitrary context $\RHO$ to begin with.
\begin{displaymath}
  \inference{?}{
    \typing{\RHO}{\PLUS{\TRUE}{\NUM{7}}}{\type{t}}
  }
\end{displaymath}
Just as before, we examine our typing rules to find one whose bottom matches
the expression that we're finding a derivation for.  In other words, it needs
to have a $\PLUS{}{}$ in it. There is only one: \RULE{p4}. Filling in its
premises yields:
\begin{displaymath}
  \inference[\RULE{p4}]{
    \typing{\RHO}{\TRUE}{\INT} &
    \typing{\RHO}{\NUM{7}}{\INT}
  }{
    \typing{\RHO}{\PLUS{\TRUE}{\NUM{7}}}{\INT}
  }
\end{displaymath}
We have replaced our variable $\type{t}$ with the concrete type \INT, because
that is the result specified by \RULE{p4}. Unfortunately, we've reached an
impasse - there are no rules that allow us to give \TRUE\ the type \INT.

\section{Types vs. type schemes}
\label{sec:types-vs.-type}

In our typing rules, we distinguish between \emph{types} and \emph{type
  schemes}. The difference is that type schemes are polymorphic -- their
variables can be instantiated to a number of different types. An important
point is that not every type variable creates polymorphism -- only those that
are listed between the $\scheme{\forall}$ and the dot. In this tutorial, they
are green.  Other type variables that are not listed after the
$\scheme{\forall}$ are simply useless, because there is no way to replace
them with concrete types.

Two of the rules, \RULE{p6} and \RULE{p8}, allow adding a $\scheme\forall$ and
any number of variables to the beginning of a type. Using these rules enables
us to create useful polymorphic types.  This distinction between the
quantified type variables and free type variables (that is, between the green
type variables and the blue type variables) is used to ensure that functions
and variables are first polymorphic \emph{after} they are defined and not in
their own bodies, according to the rules in \citet{Sestoft2012}, Section 6.2.

\section{Anatomy of a typing rule}
\label{sec:anatomy-typing-rule}

Before we continue to more complex examples, let's take a minute to examine
our typing rules to really understand what they are saying. At the top level,
a typing rule gives a list of zero or more conditions that must be satisfied
in order to make some conclusion. These conditions, called \emph{premises},
are written above the line. The conclusion is written below the line. In other
words, in the following rule \RULE{Foo}, $P_1$, $P_2$, and $P_3$ are the
premises and $C$ is the conclusion.
\begin{displaymath}
  \inference[\RULE{Foo}]{P_1 & P_2 & P_3}{C}
\end{displaymath}
Sometimes, if the premises are long, they may take more than one line:
\begin{displaymath}
  \inference[\RULE{Foo}]{P_1 & P_2 \\ P_3}{C}
\end{displaymath}
This means exactly the same thing. A typing rule means that \emph{all}
premises must be proved in order for the conclusion to also be considered
proven.

In our typing rules, we have the following things:
\begin{description}
\item[Typings] relate an environment, and expression, and a type. The typing \[\typing{\RHO}{\expr{e}}{\type{t}}\] states that $\expr{e}$ can have the type $\type{t}$ in an environment $\RHO$.
\item[Substitutions] take a scheme and replace its quantified (that is, green) variables with
  types, yielding a new type. The substitution $\SUBST{\type{\BOOL}/\scheme{\alpha}}{\fn{\fn{\scheme{\alpha}}{\scheme{\alpha}}}{\INT}}$ yields the type $\type{\fn{\fn{\BOOL}{\BOOL}}{\INT}}$.
\item[Environment lookups] check that some name is mapped to some type scheme
  in an environment. We typically use this when we expect something to have
  been added to the environment further down the tree, such as when a function
  or variable has been defined and we need to use its type. The assertion that
  $\var{x}$ in our environment $\RHO$ has type scheme $\FORALL{\alpha}{\fn{\scheme{\alpha}}{\scheme{\alpha}}}$ is written $\LOOKUP{\RHO}{\var{x}} = \FORALL{\alpha}{\fn{\scheme{\alpha}}{\scheme{\alpha}}}$. This only occurs in the rules together with substitution, which is why rule \RULE{p3} can be called the \emph{specialization} rule -- it specializes a polymorphic type scheme from the environment to a useful concrete type.
\item[Environment extensions] take an environment and add some mappings to it,
  possibly replace ones that already exist. $\WITH{\RHO}{\var{x}\mapsto\INT}$
  denotes \RHO\ extended with a mapping from $\var{x}$ to \INT.
\item[Side conditions] are pieces of English text that capture a property that
  would be to complex or take up too much space to define purely
  mathematically. In the rules given here, only \RULE{p6} and \RULE{p8} have
  side conditions that state that the type variables being quantified are not
  free. The precise definition of this is somewhat subtle and beyond the scope
  of this tutorial -- it is sufficient for you to know that you should pick
  unique names for your type variables, and then you will never need to worry
  about this condition.
\end{description}

\section{Monomorphic let-bindings}
\label{sec:monom-let-bind}

Returning to our examples, we will now see how \emph{monomorphic} functions
and let-bound variables are typed. In other words, the examples in this
section have only concrete types.

\subsection{Variables}
\label{sec:variables}


We begin with a simple let-binding. We will show that
$\LET{\var{n}}{\NUM{17}}{\PLUS{\var{n}}{\NUM{1}}}$ can have type \INT\ in any
environment.  We represent the ``in any environment'' part by simply placing
some abstract environment variable $\RHO$ to the left of the
turnstile. Because we never use the initial contents of $\RHO$, they can be
anything.

We want to show, therefore,
$\typing{\RHO}{\LET{\var{n}}{\NUM{17}}{\PLUS{\var{n}}{\NUM{1}}}}{\INT}$, using
our typing rules. We begin as before, by creating a new tree with our goal at
the bottom:
\begin{displaymath}
  \inference{?}{
    \typing{\RHO}{\LET{\var{n}}{\NUM{17}}{\PLUS{\var{n}}{\NUM{1}}}}{\INT}
  }
\end{displaymath}
We have only one rule that applies: \RULE{p6}. We cannot use \RULE{p8} because
it only applies when there are \emph{two} names following the \kwd{let}. The
first premise of \RULE{p6} states that we need to type-check the expression
being bound to $\var{n}$, and then extend our environment with the type we
discover for $\var{n}$ and typecheck the body of the \kwd{let} with this
extended environment. When we extend our environment, we also have the
possibility to create a polymorphic type, but we will not use that possibility
here. See Section \ref{sec:polym-let-bind} for examples of this. Our current
state is now:
\begin{displaymath}
  \inference[\RULE{p6}]{
    \typing{\RHO}{\NUM{17}}{\INT} &
    \typing{\WITH{\RHO}{\var{n}\mapsto \INT}}{\PLUS{\var{n}}{\NUM{1}}}{\INT}
  }{
    \typing{\RHO}{\LET{\var{n}}{\NUM{17}}{\PLUS{\var{n}}{\NUM{1}}}}{\INT}
  }
\end{displaymath}

The first premise, that $\NUM{17}$ can be an \INT, is easy to show using rule
\RULE{p1}. In the second case, the only rule that has a $\PLUS{}{}$ in the
conclusion is \RULE{p4}:
\begin{displaymath}
  \inference[\RULE{p6}]{
    \inference[\RULE{p1}]{}{\typing{\RHO}{\NUM{17}}{\INT}} &
    \inference[\RULE{p4}]{
      \typing{\WITH{\RHO}{\var{n}\mapsto \INT}}{\var{n}}{\INT} &
      \typing{\WITH{\RHO}{\var{n}\mapsto \INT}}{\NUM{1}}{\INT}
    }{
      \typing{\WITH{\RHO}{\var{n}\mapsto \INT}}{\PLUS{\var{n}}{\NUM{1}}}{\INT}
    }
  }{
    \typing{\RHO}{\LET{\var{n}}{\NUM{17}}{\PLUS{\var{n}}{\NUM{1}}}}{\INT}
  }
\end{displaymath}

The interesting case here is showing that $\var{n}$ can indeed have type
\INT. To do this, we look up $\var{n}$ in our extended environment using rule
\RULE{p3}. We can safely ignore the substitutions performed in \RULE{p3}'s
conclusion, because we don't have any $\scheme{\forall}$-bound type variables
to replace. Our final tree is then:
\begin{displaymath}
\resizebox{1.0\textwidth}{!}{
  \inference[\RULE{p6}]{
    \inference[\RULE{p1}]{}{\typing{\RHO}{\NUM{17}}{\INT}} &
    \inference[\RULE{p4}]{
      \inference[\RULE{p3}]{
        \LOOKUP{\WITH{\RHO}{\var{n}\mapsto \INT}}{\var{n}} = \INT
      }{
        \typing{\WITH{\RHO}{\var{n}\mapsto \INT}}{\var{n}}{\INT}
      } &
      \inference[\RULE{p1}]{
      }{
        \typing{\WITH{\RHO}{\var{n}\mapsto \INT}}{\NUM{1}}{\INT}
      }
    }{
      \typing{\WITH{\RHO}{\var{n}\mapsto \INT}}{\PLUS{\var{n}}{\NUM{1}}}{\INT}
    }
  }{
    \typing{\RHO}{\LET{\var{n}}{\NUM{17}}{\PLUS{\var{n}}{\NUM{1}}}}{\INT}
  }
}
\end{displaymath}

\subsection{The Fibonacci function}
\label{sec:fibonacci-function}

The Fibonacci function can be defined as follows in our little dialect of ML:
\begin{displaymath}
  \LETF{
    \var{fib}
  }{
    \var{n}
  }{
    \IF{
      \LT{\var{n}}{\NUM{1}}
    }{
      \NUM{1}
    }{
      \PLUS{
        \app{\var{fib}}{(\PLUS{\var{n}}{\NUM{-1}})}
      }{
        \app{\var{fib}}{(\PLUS{\var{n}}{\NUM{-2}})}
      }
    }
  }{
    \var{fib}
  }
\end{displaymath}
By looking at the function, we can see that it should have type
$\fn{\INT}{\INT}$ in any context. Thus, we will prove that
\begin{displaymath}
  \typing{
    \RHO
  }{
    \LETF{
      \var{fib}
    }{
      \var{n}
    }{
      \IF{
        \LT{\var{n}}{\NUM{1}}
      }{
        \NUM{1}
      }{
        \PLUS{
          \app{\var{fib}}{(\PLUS{\var{n}}{\NUM{-1}})}
        }{
          \app{\var{fib}}{(\PLUS{\var{n}}{\NUM{-2}})}
        }
      }
    }{
      \var{fib}
    }
  }{
    \fn{\INT}{\INT}
  }
\end{displaymath}

We are beginning to approach the maximum width that can be displayed on a
printed page, so the derivation will be constructed in manageable pieces and
presented whole at the end.  Because we are defining a function, we must use
\RULE{p8}. This rule has two premises that we need to prove:
\\\noindent
\begin{displaymath}
  \typing{
    \WITH{\RHO}{\var{n} \mapsto \type{t_x}, \var{fib} \mapsto \type{\fn{t_x}{t_r}}}
  }{
    \IF{
      \LT{\var{n}}{\NUM{1}}
    }{
      \NUM{1}
    }{
      \PLUS{
        \app{\var{fib}}{(\PLUS{\var{n}}{\NUM{-1}})}
      }{
        \app{\var{fib}}{(\PLUS{\var{n}}{\NUM{-2}})}
      }
    }
  }{\type{t_r}}
\end{displaymath}
and
\begin{displaymath}
\typing{\WITH{\RHO}{\var{fib} \mapsto \fn{\type{t_x}}{\type{t_r}}}}{\var{fib}}{\fn{\INT}{\INT}}
\end{displaymath}

Proving them requires us to do two things: we must find appropriate types to
fill in $\type{t_x}$ and $\type{t_r}$, and we must construct derivation trees
that demonstrate that these values are indeed valid types. By looking at the
above statements, it is straightforward to see that both $\type{t_x}$ and
$\type{t_r}$ need to be \INT, because $\var{n}$ is compared to an integer and
the result of the function is used in an addition. Thus, the second premise
will be straightforward, requiring only a use of \RULE{p3} to look up
\var{fib}'s type in the context.

To construct the derivation for the first premise, we use \RULE{p7}, which now
requires us to show the following:
\begin{itemize}
\item $
  \typing{
    \WITH{\RHO}{\var{n} \mapsto \type{\INT}, \var{fib} \mapsto \type{\fn{\INT}{\INT}}}
  }{
    \LT{\var{n}}{\NUM{1}}
  }{
    \BOOL
  }
$
\item $
  \typing{
    \WITH{\RHO}{\var{n} \mapsto \type{\INT}, \var{fib} \mapsto \type{\fn{\INT}{\INT}}}
  }{
    \NUM{1}
  }{
    \INT
  }
$
\item $
  \typing{
    \WITH{\RHO}{\var{n} \mapsto \type{\INT}, \var{fib} \mapsto \type{\fn{\INT}{\INT}}}
  }{
    \PLUS{
      \app{\var{fib}}{(\PLUS{\var{n}}{\NUM{-1}})}
    }{
      \app{\var{fib}}{(\PLUS{\var{n}}{\NUM{-2}})}
    }
  }{
    \INT
  }
$
\end{itemize}

We proceed with the derivations for these typings.  In the interests of saving
space, repeated copies of the same environment extensions have been replaced
by ``$\cdots$'' when they were not directly relevant to the derivation and
could be seen elsewhere.  The first derivation checks that both arguments to
the \LT{}{} are integers, looking up \var{n} in the environment:
\begin{displaymath}
\resizebox{1.0\textwidth}{!}{
  \inference[\RULE{p5}]{
    \inference[\RULE{p3}]{
      \LOOKUP{
        \WITH{\RHO}{\var{n} \mapsto \type{\INT}, \var{fib} \mapsto \type{\fn{\INT}{\INT}}}
      }{
        \var{n}
      } = \INT
    }{
      \typing{
        \WITH{\RHO}{\var{n} \mapsto \type{\INT}, \var{fib} \mapsto \type{\fn{\INT}{\INT}}}
      }{
        \var{n}
      }{
        \INT
      }
    } &
    \inference[\RULE{p1}]{}{
      \typing{
        \WITH{\RHO}{\cdots}
      }{
        \NUM{1}
      }{
        \INT
      }
     }
  }{
  \typing{
    \WITH{\RHO}{\var{n} \mapsto \type{\INT}, \var{fib} \mapsto \type{\fn{\INT}{\INT}}}
  }{
    \LT{\var{n}}{\NUM{1}}
  }{
    \BOOL
  }
}
}
\end{displaymath}

The base case for the recursion is straightforward to check. We use \RULE{p1},
because it is an integer constant.
\begin{displaymath}
  \inference[\RULE{p1}]{
  }{
  \typing{
    \WITH{\RHO}{\var{n} \mapsto \type{\INT}, \var{fib} \mapsto \type{\fn{\INT}{\INT}}}
  }{
    \NUM{1}
  }{
    \INT
  }
  }
\end{displaymath}

To check our recursive call, we need to check that both arguments to \PLUS{}{}
can have type $\INT$ in order to use \RULE{p4}. Here, we see only the first
one, as the second can be constructed by replacing all instances of \NUM{-1}
with \NUM{-2}.
\begin{displaymath}
\resizebox{1.10\textwidth}{!}{
  \inference[\RULE{p9}]{
    \inference[\RULE{p3}]{
      \LOOKUP{\WITH{\RHO}{\var{n} \mapsto \type{\INT}, \var{fib} \mapsto \type{\fn{\INT}{\INT}}}}{\var{fib}} = \fn{\INT}{\INT}
    }{
      \typing{
        \WITH{\RHO}{\var{n} \mapsto \type{\INT}, \var{fib} \mapsto \type{\fn{\INT}{\INT}}}
      }{
        \var{fib}
      }{
        \fn{\INT}{\INT}
      }
    } &
    \inference[\RULE{p4}]{
      \vdots \qquad\qquad &\qquad\qquad \vdots
    }{
      \typing{\WITH{\RHO}{\var{n} \mapsto \type{\INT},\cdots}}{\PLUS{\var{n}}{\NUM{-1}}}{\INT}
    }
  }{
    \typing{
      \WITH{\RHO}{\var{n} \mapsto \type{\INT}, \var{fib} \mapsto \type{\fn{\INT}{\INT}}}
    }{
      \app{\var{fib}}{(\PLUS{\var{n}}{\NUM{-1}})}
    }{
      \INT
    }
  }
}
\end{displaymath}

The full derivation can be seen in Figure \ref{fig:fib}.

\begin{sidewaysfigure}
  \begin{align*}
    \text{Condition:} \quad &\inference[\RULE{p5}]{
          \inference[\RULE{p3}]{
            \LOOKUP{
              \WITH{\RHO}{\var{n} \mapsto \type{\INT}, \var{fib} \mapsto \type{\fn{\INT}{\INT}}}
            }{
              \var{n}
            } = \INT
          }{
            \typing{
              \WITH{\RHO}{\var{n} \mapsto \type{\INT}, \var{fib} \mapsto \type{\fn{\INT}{\INT}}}
            }{
              \var{n}
            }{
              \INT
            }
          } &
          \inference[\RULE{p1}]{}{
            \typing{
              \WITH{\RHO}{\cdots}
            }{
              \NUM{1}
            }{
              \INT
            }
          }
        }{
          \typing{
            \WITH{\RHO}{\var{n} \mapsto \type{\INT}, \var{fib} \mapsto \type{\fn{\INT}{\INT}}}
          }{
            \LT{\var{n}}{\NUM{1}}
          }{
            \BOOL
          }
        }\\\\
    \text{Recursive:} \quad &\inference[\RULE{p9}]{
      \inference[\RULE{p3}]{
        \LOOKUP{\WITH{\RHO}{\var{n} \mapsto \type{\INT}, \var{fib} \mapsto \type{\fn{\INT}{\INT}}}}{\var{fib}} = \fn{\INT}{\INT}
      }{
        \typing{
          \WITH{\RHO}{\var{n} \mapsto \type{\INT}, \var{fib} \mapsto \type{\fn{\INT}{\INT}}}
        }{
          \var{fib}
        }{
          \fn{\INT}{\INT}
        }
      } & \inference[\RULE{p4}]{
        \inference[\RULE{p3}]{
          \LOOKUP{\WITH{\RHO}{\var{n} \mapsto \type{\INT}, \var{fib} \mapsto \type{\fn{\INT}{\INT}}}}{\var{n}} = \INT
        }{
          \typing{
            \WITH{\RHO}{\var{n} \mapsto \type{\INT}, \var{fib} \mapsto \type{\fn{\INT}{\INT}}}
          }{
            \var{n}
          }{
            \INT
          }
        }
        &
        \inference[\RULE{p1}]{}{
          \typing{
            \WITH{\RHO}{\cdots}
          }{
            \NUM{-1}
          }{
            \INT
          }
        }
      }{
        \typing{
          \WITH{\RHO}{\var{n} \mapsto \type{\INT}, \var{fib} \mapsto \type{\fn{\INT}{\INT}}}
        }{
          \PLUS{\var{n}}{\NUM{-1}}
        }{
          \INT
        }
      }
    }{
      \typing{
        \WITH{\RHO}{\var{n} \mapsto \type{\INT}, \var{fib} \mapsto \type{\fn{\INT}{\INT}}}
      }{
        \app{\var{fib}}{(\PLUS{\var{n}}{\NUM{-1}})}
      }{
        \INT
      }
    }\\\\
    \text{Body derivation:} \quad& \inference[\RULE{p7}] {
      \text{(Condition)}
         &
        \inference[\RULE{p1}]{}{
          \typing{\WITH{\RHO}{\cdots}}{\NUM{1}}{\INT}
        } &
        \inference[\RULE{p4}]{
          \text{(Recursive)}\quad & \quad\text{(Recursive with \NUM{-2})}
        }{
          \typing{
            \WITH{\RHO}{\var{n} \mapsto \type{\INT}, \var{fib} \mapsto \type{\fn{\INT}{\INT}}}
          }{
            \PLUS{
              \app{\var{fib}}{(\PLUS{\var{n}}{\NUM{-1}})}
            }{
              \app{\var{fib}}{(\PLUS{\var{n}}{\NUM{-2}})}
            }
          }{
            \INT
          }}
       }{
        \typing{
          \WITH{\RHO}{\var{n} \mapsto \type{\INT}, \var{fib} \mapsto \type{\fn{\INT}{\INT}}}
        }{
          \IF{
            \LT{\var{n}}{\NUM{1}}
          }{
            \NUM{1}
          }{
            \PLUS{
              \app{\var{fib}}{(\PLUS{\var{n}}{\NUM{-1}})}
            }{
              \app{\var{fib}}{(\PLUS{\var{n}}{\NUM{-2}})}
            }
          }
        }{\type{\INT}}
      }
  \end{align*}
  \vspace{1cm}
  \begin{displaymath}
    \inference[\RULE{p8}]{
      \text{(Body derivation)}
       &
      \inference[]{}{
        \typing{\WITH{\RHO}{\var{fib} \mapsto \fn{\type{\INT}}{\type{\INT}}}}{\var{fib}}{\fn{\INT}{\INT}}
      }
    }{
      \typing{
        \RHO
      }{
        \LETF{
          \var{fib}
        }{
          \var{n}
        }{
          \IF{
            \LT{\var{n}}{\NUM{1}}
          }{
            \NUM{1}
          }{
            \PLUS{
              \app{\var{fib}}{(\PLUS{\var{n}}{\NUM{-1}})}
            }{
              \app{\var{fib}}{(\PLUS{\var{n}}{\NUM{-2}})}
            }
          }
        }{
          \var{fib}
        }
      }{
        \fn{\INT}{\INT}
      }
    }
  \end{displaymath}
  \caption{The full typing derivation for the Fibonacci function. Sub-derivations are named and written separately.\label{fig:fib}}
\end{sidewaysfigure}

\section{Polymorphic let-bindings}
\label{sec:polym-let-bind}

In the preceding examples, we have defined variables and functions through
\kwd{let}-bindings.  In these examples, we've added new names to contexts and
we've looked them up. However, we have not yet defined any polymorphic values
or functions, so the substitution step in the lookup has been trivial. In this
section, we'll explore how polymorphic functions are defined and how their
type derivations are constructed. In doing so, we'll actually use the
substitution feature.

\subsection{The identity function}
\label{sec:identity-function}

The identity function, written $\LETF{\var{id}}{\var{x}}{\var{x}}{\var{id}}$,
can have any number of different types.  Let's begin by showing that it can
have the type $\fn{\INT}{\INT}$ in any environment. We begin by using an
unspecified environment \RHO, our function, and our desired type:
\begin{displaymath}
  \inference[]{
    ?
  }{
    \typing{\RHO}{\LETF{\var{id}}{\var{x}}{\var{x}}{\var{id}}}{\fn{\INT}{\INT}}
  }
\end{displaymath}
Applying \RULE{p8}, we get two new premises.  The body of our function, in
this case \var{x}, must be typable as some result type $\type{t_r}$, given
that $\var{id}$ can be a function from $\var{x}$'s type to that result
type. Additionally, if we then attach a type scheme quantifier over some type
variables, we then need to show that the body has the correct type for the
whole let (in this case \type{\fn{\INT}{\INT}}). Our tree is as follows:
\begin{displaymath}
\resizebox{1.10\textwidth}{!}{
  \inference[\RULE{p8}]{
    \inference[]{?}{
      \typing{
        \WITH{\RHO}{\var{id}\mapsto\type{\fn{t_x}{t_r}},
          \var{x}\mapsto\type{t_x}}
      }{
        \var{x}
      }{
        \type{t_r}
      }
    } &
    \inference[]{?}{
      \typing{
        \WITH{\RHO}{\var{id}\mapsto\FORALL{\ALPHAS}{\fn{\type{t_x}}{\type{t_r}}}}
      }{
        \var{id}
      }{
        \fn{\INT}{\INT}
      }
    }
  }{
    \typing{\RHO}{\LETF{\var{id}}{\var{x}}{\var{x}}{\var{id}}}{\fn{\INT}{\INT}}
  }
}
\end{displaymath}
To complete the derivation, we must figure out how to fill out $\type{t_x}$,
$\type{t_r}$, and $\scheme{\ALPHAS}$ such that we can use \RULE{p3} for both
cases.

The first observation to be made is that $\type{t_x}$ has to be the same as
$\type{t_r}$. This is because \var{id} returns its argument unchanged. The
other observation is that we have no other restrictions on them, because no
operations are applied to \var{x} in \var{id}'s definition. We are free to
choose \emph{any} type here.  Thus, we can pick a type variable
$\type{\alpha}$. We now have:
\begin{displaymath}
\resizebox{1.10\textwidth}{!}{
  \inference[\RULE{p8}]{
    \inference[]{?}{
      \typing{
        \WITH{\RHO}{\var{id}\mapsto\type{\fn{\alpha}{\alpha}},
          \var{x}\mapsto\type{\alpha}}
      }{
        \var{x}
      }{
        \type{\alpha}
      }
    } &
    \inference[]{?}{
      \typing{
        \WITH{\RHO}{\var{id}\mapsto\FORALL{\ALPHAS}{\fn{\type{\alpha}}{\type{\alpha}}}}
      }{
        \var{id}
      }{
        \fn{\INT}{\INT}
      }
    }
  }{
    \typing{\RHO}{\LETF{\var{id}}{\var{x}}{\var{x}}{\var{id}}}{\fn{\INT}{\INT}}
  }
}
\end{displaymath}

Secondly, we can replace the $\scheme{\ALPHAS}$ with the single variable
$\scheme{\alpha}$, as it is the only type variable in the right-hand part of
the type scheme. Because the variable $\type{\alpha}$ is now bound by the
scheme, I change its color to green.
\begin{displaymath}
\resizebox{1.10\textwidth}{!}{
  \inference[\RULE{p8}]{
    \inference[]{?}{
      \typing{
        \WITH{\RHO}{\var{id}\mapsto\type{\fn{\alpha}{\alpha}},
          \var{x}\mapsto\type{\alpha}}
      }{
        \var{x}
      }{
        \type{\alpha}
      }
    } &
    \inference[]{?}{
      \typing{
        \WITH{\RHO}{\var{id}\mapsto\FORALL{\scheme{\alpha}}{\fn{\scheme{\alpha}}{\scheme{\alpha}}}}
      }{
        \var{id}
      }{
        \fn{\INT}{\INT}
      }
    }
  }{
    \typing{\RHO}{\LETF{\var{id}}{\var{x}}{\var{x}}{\var{id}}}{\fn{\INT}{\INT}}
  }
}
\end{displaymath}
We're almost done.  What's left is to use \RULE{p3} to look up our \var{x} and
\var{id} in the environments.  Here, it's important to look at the structure
of \RULE{p3} carefully: the \scheme{\ALPHAS} being substituted on the bottom
are the $\scheme{\alpha}$s between the \scheme{\forall} and the dot, not just any
arbitrary variables in the type. In this tutorial, the variables that can be
substituted are always green.
\begin{displaymath}
\resizebox{1.10\textwidth}{!}{
  \inference[\RULE{p8}]{
    \inference[\RULE{p3}]{
        \LOOKUP{
          \WITH{\RHO}{\var{id}\mapsto\type{\fn{\alpha}{\alpha}}, \var{x}\mapsto\type{\alpha}}
        }{\var{x}} = \type{\alpha}
    }{
      \typing{
        \WITH{\RHO}{\var{id}\mapsto\type{\fn{\alpha}{\alpha}}, \var{x}\mapsto\type{\alpha}}
      }{
        \var{x}
      }{
        \type{\alpha}
      }
    } &
    \inference[]{
      \LOOKUP{
        \WITH{\RHO}{\var{id}\mapsto\FORALL{\scheme{\alpha}}{\fn{\scheme{\alpha}}{\scheme{\alpha}}}}
      }{\var{id}} = \FORALL{\alpha}{\fn{\scheme{\alpha}}{\scheme{\alpha}}}
    }{
      \typing{
        \WITH{\RHO}{\var{id}\mapsto\FORALL{\scheme{\alpha}}{\fn{\scheme{\alpha}}{\scheme{\alpha}}}}
      }{
        \var{id}
      }{
        \fn{\INT}{\INT}
      }
    }
  }{
    \typing{\RHO}{\LETF{\var{id}}{\var{x}}{\var{x}}{\var{id}}}{\fn{\INT}{\INT}}
  }
}
\end{displaymath}
The leftmost lookup performs no substitutions. The rightmost is justified
because the $[\INT/\scheme{\alpha}]\fn{\scheme{\alpha}}{\scheme{\alpha}}$ from
the bottom of \RULE{p3} is equivalent to $\fn{\INT}{\INT}$.

\subsection{Applying \var{id} to itself}
\label{sec:apply-varid-itself}

In the previous example, we used polymorphism to allow \var{id} to be used
with integers.  The same argument could have been used for any type - we would
simply replace both \INT s with the new type in our derivation. However,
polymorphism gives us more power than this.  The same function can be used
with \emph{different} types in the same expression. For an example of this, we
will apply the identity function to itself, producing an identity function on
\BOOL s.

Formally speaking, we will show that:
\begin{displaymath}
  \typing{\RHO}{\LETF{\var{id}}{\var{x}}{\var{x}}{\app{\var{id}}{\var{id}}}}{\fn{\BOOL}{\BOOL}}
\end{displaymath}

As usual, we place our goal at the bottom of a new tree.  The outermost layer
is a function let, so we use \RULE{p8}.  The first premise is solved just as
in the last example, using \RULE{p3}.
\begin{displaymath}
\resizebox{1.10\textwidth}{!}{
  \inference[\RULE{p8}]{
    \inference[\RULE{p3}]{
      \LOOKUP{
        \WITH{\RHO}{\var{id}\mapsto \fn{\type{\alpha}}{\type{\alpha}}, \var{x} \mapsto \type{\alpha}}
      }{\var{x}}
      = \type{\alpha}
    }{
      \typing{
        \WITH{\RHO}{\var{id}\mapsto \fn{\type{\alpha}}{\type{\alpha}}, \var{x} \mapsto \type{\alpha}}
      }{
        \var{x}
      }{
        \type{\alpha}
      }
    }
    &
    \inference{?}{
      \typing{
        \WITH{\RHO}{\var{id}\mapsto\FORALL{\scheme{\alpha}}{\fn{\scheme{\alpha}}{\scheme{\alpha}}}}
      }{
        \app{\var{id}}{\var{id}}
      }{
        \fn{\BOOL}{\BOOL}
      }
    }
  }{
    \typing{\RHO}{\LETF{\var{id}}{\var{x}}{\var{x}}{\app{\var{id}}{\var{id}}}}{\fn{\BOOL}{\BOOL}}
  }
}
\end{displaymath}

Let's focus on the second premise of \RULE{p8} to save space:
\begin{displaymath}
      \inference{?}{
      \typing{
        \WITH{\RHO}{\var{id}\mapsto\FORALL{\scheme{\alpha}}{\fn{\scheme{\alpha}}{\scheme{\alpha}}}}
      }{
        \app{\var{id}}{\var{id}}
      }{
        \fn{\BOOL}{\BOOL}
      }
    }
\end{displaymath}
Because it is a function call, we use \RULE{p9}:
\begin{displaymath}
  \inference[\RULE{p9}]{
    \inference{?}{
      \typing{
        \WITH{\RHO}{\var{id}\mapsto\FORALL{\scheme{\alpha}}{\fn{\scheme{\alpha}}{\scheme{\alpha}}}}
      }{ \var{id} }{\fn{\type{t_x}}{\type{t_r}}}
    }
    &
    \inference{?}{
      \typing{
        \WITH{\RHO}{\var{id}\mapsto\FORALL{\scheme{\alpha}}{\fn{\scheme{\alpha}}{\scheme{\alpha}}}}
      }{ \var{id} }{\type{t_x}}
    }
  }{
      \typing{
        \WITH{\RHO}{\var{id}\mapsto\FORALL{\scheme{\alpha}}{\fn{\scheme{\alpha}}{\scheme{\alpha}}}}
      }{
        \app{\var{id}}{\var{id}}
      }{
        \fn{\BOOL}{\BOOL}
      }
    }
\end{displaymath}
We need to demonstrate two different types for \var{id}! This is because one
instance if \var{id} is the one that we will later use on our Booleans, while
the other is actually being used on the \fn{\BOOL}{\BOOL} function
\var{id}. Therefore, both $\type{t_x}$ and $\type{t_r}$ need to be
$\fn{\BOOL}{\BOOL}$. Splitting the premises in the interest of space, we get:
\begin{displaymath}
    \inference{?}{
      \typing{
        \WITH{\RHO}{\var{id}\mapsto\FORALL{\scheme{\alpha}}{\fn{\scheme{\alpha}}{\scheme{\alpha}}}}
      }{ \var{id} }{\fn{\type{(\fn{\BOOL}{\BOOL})}}{\type{(\fn{\BOOL}{\BOOL})}}}
    }
\end{displaymath} and
\begin{displaymath}
    \inference{?}{
      \typing{
        \WITH{\RHO}{\var{id}\mapsto\FORALL{\scheme{\alpha}}{\fn{\scheme{\alpha}}{\scheme{\alpha}}}}
      }{ \var{id} }{\type{\fn{\BOOL}{\BOOL}}}
    }
\end{displaymath}

The second premise is straightforward. We simply apply \RULE{p3}, substituting
\BOOL\ for $\scheme{\alpha}$:
\begin{displaymath}
    \inference[\RULE{p3}]{
      \LOOKUP{\WITH{\RHO}{\var{id}\mapsto\FORALL{\scheme{\alpha}}{\fn{\scheme{\alpha}}{\scheme{\alpha}}}}}{\var{id}} = \FORALL{\scheme{\alpha}}{\fn{\scheme{\alpha}}{\scheme{\alpha}}}
    }{
      \typing{
        \WITH{\RHO}{\var{id}\mapsto\FORALL{\scheme{\alpha}}{\fn{\scheme{\alpha}}{\scheme{\alpha}}}}
      }{ \var{id} }{\type{\fn{\BOOL}{\BOOL}}}
    }
\end{displaymath}
This is acceptable because $\fn{\BOOL}{\BOOL} = [\BOOL / \scheme{\alpha}]\fn{\scheme{\alpha}}{\scheme{\alpha}}$.

Proving first premise is almost the same -- the only difference is our choice
of substitution for $\scheme{\alpha}$. Here, we perform the substitution $[\fn{\BOOL}{\BOOL} / \scheme{\alpha}]\fn{\scheme{\alpha}}{\scheme{\alpha}}$.
\begin{displaymath}
    \inference[\RULE{p3}]{
      \LOOKUP{\WITH{\RHO}{\var{id}\mapsto\FORALL{\scheme{\alpha}}{\fn{\scheme{\alpha}}{\scheme{\alpha}}}}}{\var{id}} = \FORALL{\scheme{\alpha}}{\fn{\scheme{\alpha}}{\scheme{\alpha}}}
    }{
      \typing{
        \WITH{\RHO}{\var{id}\mapsto\FORALL{\scheme{\alpha}}{\fn{\scheme{\alpha}}{\scheme{\alpha}}}}
      }{ \var{id} }{\type{\fn{(\fn{\BOOL}{\BOOL})}{(\fn{\BOOL}{\BOOL})}}}
    }
\end{displaymath}

By instantiating our type schemes differently, we can get a variety of types for the same term.
\appendix
\section{Formal rules}
\label{sec:formal-rules}

\subsection*{Types}
\grammar{\type{t}}{Types}
\prod{\INT}{Integers}\\
\prod{\BOOL}{Booleans}\\
\prod{\type{\fn{t}{t}}}{Functions}\\
\prod{\type{\alpha}\mid\type{\beta}\mid\type{\gamma}\mid\type{\delta}\mid\ldots}{Type variables}\\
\endgrammar

\subsection*{Type schemes}
\grammar{\scheme{\sigma}}{Typing schemes}
\prod{\type{t}}{Simple type with no parameters}\\
\prod{\FORALL{\ALPHAS}{\type{t}}}{\type{t} with \scheme{\ALPHAS} as parameters}
\endgrammar

\subsection*{Expressions}
\grammar{\expr{e}}{Expressions}
\prod{\expr{x} \mid \expr{y}\mid\expr{z}\mid\expr{f}}{Variables}\\
\prod{\textcolor{term}{i}}{Integer literals}\\
\prod{\PLUS{\expr{e}}{\expr{e}}}{Addition}\\
\prod{\LT{\expr{e}}{\expr{e}}}{Less than}\\
\prod{\TRUE}{True}\\
\prod{\FALSE}{False}\\
\prod{\app{\expr{e}}{\expr{e}}}{Function application}\\
\prod{\IF{\expr{e}}{\expr{e}}{\expr{e}}}{Conditional expression}\\
\prod{\LET{\expr{x}}{\expr{e}}{\expr{e}}}{Variable binding}\\
\prod{\LETF{\expr{x}}{\expr{x}}{\expr{e}}{\expr{e}}}{Variable binding}\\
\endgrammar

\subsection*{Contexts}
\grammar{\RHO}{Typing contexts}
\prod{\EMPTY}{The empty context}\\
\prod{\WITH{\RHO}{\expr{x_1} \mapsto \scheme{\sigma_1}, \ldots, \expr{x_n} \mapsto \scheme{\sigma_n}}}{\RHO\ extended with new name/scheme pairs}
\endgrammar

The symbol $\mapsto$ is pronounced ``maps to''.

\subsection*{Typing Rules}
These typing rules are taken from \citet[p. 98]{Sestoft2012}, and have been
slightly reformatted to fit the page and match the color scheme of the
tutorial.

\begin{displaymath}
  \inference[\RULE{p1}]{}{
    \typing{\RHO}{\textcolor{term}{i}}{\INT}
  }
\end{displaymath}

\begin{displaymath}
  \inference[\RULE{p2}]{}{
    \typing{\RHO}{\textcolor{term}{b}}{\BOOL}
  }
\end{displaymath}

% \begin{displaymath}
%   \inference[\RULE{p2a}]{}{
%     \typing{\RHO}{\TRUE}{\BOOL}
%   }
% \end{displaymath}

% \begin{displaymath}
%   \inference[\RULE{p2b}]{}{
%     \typing{\RHO}{\FALSE}{\BOOL}
%   }
% \end{displaymath}

\begin{displaymath}
  \inference[\RULE{p3}]{
    \LOOKUP{\RHO}{\var{f}} = \FORALL{\alpha_1}{\type{t}}
  }{
    \typing{\RHO}{\var{f}}{
      [\type{t_1}/\scheme{\alpha_1}, \ldots, \type{t_n}/\scheme{\alpha_n}]\type{t}
    }
  }
\end{displaymath}

\begin{displaymath}
  \inference[\RULE{p4}]{
    \typing{\RHO}{\expr{e_1}}{\INT} &
    \typing{\RHO}{\expr{e_2}}{\INT}
  }{
    \typing{\RHO}{\PLUS{\expr{e_1}}{\expr{e_2}}}{\INT}
  }
\end{displaymath}

\begin{displaymath}
  \inference[\RULE{p5}]{
    \typing{\RHO}{\expr{e_1}}{\INT} &
    \typing{\RHO}{\expr{e_2}}{\INT}
  }{
    \typing{\RHO}{\LT{\expr{e_1}}{\expr{e_2}}}{\BOOL}
  }
\end{displaymath}

\begin{displaymath}
  \inference[\RULE{p6}]{
    \typing{\RHO}{\expr{e_r}}{\type{t_r}} &
    \typing{
      \WITH{\RHO}{\black{x} \mapsto \FORALL{\ALPHAS}{\type{t_r}}}
    }{
      \expr{e_b}
    }{
      \type{t}
    } \\
    \text{$\scheme{\ALPHAS}$ not free in $\RHO$}
  }{
    \typing{\RHO}{\LET{x}{\expr{e_r}}{\expr{e_b}}}{\type{t}}
  }
\end{displaymath}


\begin{displaymath}
  \inference[\RULE{p7}]{
    \typing{\RHO}{\expr{e_1}}{\BOOL} &
    \typing{\RHO}{\expr{e_2}}{\type{t}} &
    \typing{\RHO}{\expr{e_3}}{\type{t}}
  }{
    \typing{\RHO}{\IF{\expr{e_1}}{\expr{e_2}}{\expr{e_3}}}{\type{t}}
  }
\end{displaymath}

\begin{displaymath}
  \inference[\RULE{p8}]{
    \typing{
      \WITH{\RHO}{\expr{x} \mapsto \type{t_x}, \expr{f}\mapsto \type{\fn{t_x}{t_r}}}
    }{
      \expr{e_r}
    }{
      \type{t_r}
    } &
    \typing{
      \WITH{\RHO}{\expr{f}\mapsto\FORALL{\ALPHAS}{\type{\fn{t_x}{t_r}}}}
    }{
      \expr{e_b}
    }{
      \type{t}
    }\\
    \text{$\scheme{\ALPHAS}$ not free in $\RHO$}
  }{
    \typing{\RHO}{\LETF{\expr{f}}{\expr{x}}{\expr{e_r}}{\expr{e_b}}}{\type{t}}
  }
\end{displaymath}

\begin{displaymath}
  \inference[\RULE{p9}]{
    \typing{\RHO}{\expr{e_1}}{\type{\fn{t_x}{t_r}}} &
    \typing{\RHO}{\expr{e_2}}{\type{t_x}}
  }{
    \typing{\RHO}{\expr{\app{e_1}{e_2}}}{\type{t_r}}
  }
\end{displaymath}

\section*{Acknowledgments}
I would like to thank Hannes Mehnert and Peter Sestoft for comments on drafts
of this tutorial as well as David Georg Korczynski for catching an error.

\bibliographystyle{plainnat}
\bibliography{type-rule-tutorial}
\end{document}
