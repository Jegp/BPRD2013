\documentclass[a4paper, titlepage]{article}

\usepackage{graphicx} % Required to insert images
\usepackage{natbib}
\usepackage{rotating}

\usepackage{amsmath}
\usepackage{tabularx}
\usepackage[utf8]{inputenc}

\begin{document}

\title{Week 40}
\author{Sigurt Bladt Dinesen \\sidi{@}itu.dk \and Jens Egholm Pedersen \\jegp{@}itu.dk}
\maketitle
\begin{verbatim}
 1 > fromFile "ex1.c";;
 2 val it : Absyn.program =
 3   Prog
 4     [Fundec
 5        (null,"main",[(TypI, "n")],
 6         Block
 7           [Stmt
 8              (While
 9                (Prim2 (">",Access (AccVar "n"),CstI 0),
10                 Block
11                  [Stmt (Expr (Prim1 ("printi",Access (AccVar "n"))));
12                   Stmt
13                     (Expr
14                       (Assign
15                          (AccVar "n",Prim2 ("-",Access (AccVar "n"),CstI 1))))]));
16            Stmt (Expr (Prim1 ("printc",CstI 10)))])]
\end{verbatim}

\quad

\begin{tabular}{l p{12cm}}
  \textbf{Declerations} & \texttt{Fundec} declares the \texttt{main} 
  function with no return-type, name "main" and the integer 
  (\texttt{TypI}) \texttt{i} as the only parameter. \\
  \textbf{Statements} & The first statement is the Block-statement 
  in line 6 followed by a while-statement which also contains three
  statements: One that prints the variable $n$ (in line 11), 
  one that assigns $n - 1$ to $n$ and one that decrements the 
  variable $n$. \\
  & Lastly in line 16 a statement which prints the ASCII character 
  for a line break. \\
  \textbf{Types} & Types \texttt{TypI} for the paramenter of the 
  \texttt{main} method (line 5). \\
  \textbf{Expressions} & In line 11 we call the method 
  \texttt{printi} using the variable \texttt{n}. In line 13 we assign 
  a new value to the variable \texttt{n}. Line 16 contains an 
  expression where the method \texttt{printc} is called using the 
  constant $10$.
\end{tabular}

\section*{7.2}


\end{document}
