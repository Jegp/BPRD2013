\documentclass[a4paper]{article}

\usepackage{graphicx} % Required to insert images
\usepackage{natbib}
\usepackage{rotating}

\usepackage{amsmath}
\usepackage{tabularx}
\usepackage[utf8]{inputenc}

\begin{document}

\section*{7.1}
\begin{verbatim}
 1 > fromFile "ex1.c";;
 2 val it : Absyn.program =
 3   Prog
 4     [Fundec
 5        (null,"main",[(TypI, "n")],
 6         Block
 7           [Stmt
 8              (While
 9                 (Prim2 (">",Access (AccVar "n"),CstI 0),
10                  Block
11                    [Stmt (Expr (Prim1 ("printi",Access (AccVar "n"))));
12                     Stmt
13                       (Expr
14                          (Assign
15                             (AccVar "n",Prim2 ("-",Access (AccVar "n"),CstI 1))))]));
16            Stmt (Expr (Prim1 ("printc",CstI 10)))])]
\end{verbatim}

\quad

\begin{tabular}{l l}
  \textbf{Declerations} & \texttt{Fundec} declares the \texttt{main} function with no return-type, name "main" and \\
                        & the integer (\texttt{TypI}) \texttt{i} as the only parameter. \\
  \textbf{Statements} & The first statement is the while-statement in line 7 which also contains \\
                      & two statements: One that prints the variable \texttt{n} (in line 11) and one that \\
                      & decrements the variable \texttt{n}. Lastly in line 16 a statement which prints the \\
                      & ASCII character for a line break. \\
  \textbf{Types} & Types \texttt{TypI} for the paramenter of the \texttt{main} method (line 5). \\
  \textbf{Expressions} & In line 11 we call the method \texttt{printi} using the variable \texttt{n}. In line 13 we \\
                       & assign a new value to the variable \texttt{n}. Line 16 contains an expression where the \\
                       & method \texttt{printc} is called using the constnat $10$.
\end{tabular}

\section*{7.2}


\end{document}
