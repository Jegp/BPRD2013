\documentclass[a4paper, titlepage]{article}

\usepackage{graphicx} % Required to insert images
\usepackage{natbib}
\usepackage{listings}

\usepackage{amsmath}
\usepackage{tabularx}
\usepackage[utf8]{inputenc}

\begin{document}

\title{Week 44}
\author{Sigurt Bladt Dinesen \\sidi{@}itu.dk 
  \and Jens Egholm Pedersen \\jegp{@}itu.dk}
\maketitle

Note to  Exercises.fs
\section*{11.1}
\subsection*{11.1(ii)}
Calling the continuation based \emph{lenc xs : int list k : int -> int : int}
function with a function \emph{f = (fun x -> 2*x)} will return twice the list
length, as the first given continuation is the last to be executed. (otherwise
it would have returned \emph{length + f (0 * 2)}, i.e. the list's length.

\subsection*{11.1(iii)}
\emph{lenc} and \emph{leni} are essentially the same funtion. \emph{lenc}
represents the accumulating parameter of \emph{leni} by a chain of functions,
build and passed on in each recursive call. There exists a general
transformation from iterative functions (such as \emph{len} and \emph{leni}) to
an equivalent continuation based function.
\end{document}
