\documentclass[a4paper, titlepage]{article}

\usepackage{graphicx} % Required to insert images
\usepackage{natbib}
\usepackage{listings}

\usepackage{amsmath}
\usepackage{tabularx}
\usepackage[utf8]{inputenc}

\begin{document}

\title{Week 42}
\author{Sigurt Bladt Dinesen \\sidi{@}itu.dk \and Jens Egholm Pedersen \\jegp{@}itu.dk}
\maketitle
\section*{9.1}
\subsection*{9.1(i) Selsort.il}
\begin{lstlisting}[numbers=left, firstnumber=0, title=Selsort.il]
// method line 2
.method public static hidebysig 
       default void SelectionSort (int32[] arr)  cil managed 
{
// Method begins at RVA 0x20c0
// Code size 72 (0x48)
.maxstack 4
.locals init (
	int32	V_0,
	int32	V_1,
	int32	V_2,
	int32	V_3)
IL_0000:  ldc.i4.0                //push 4byte int 0
IL_0001:  stloc.0                 //store it locally
IL_0002:  br IL_003e              //jump to for guard

IL_0007:  ldloc.0                 //for body, load i
IL_0008:  stloc.1                 //store "to least"
IL_0009:  ldloc.0                 //load i again
IL_000a:  ldc.i4.1
IL_000b:  add 
IL_000c:  stloc.2                 //store i+1 (as j) for the inner loop
IL_000d:  br IL_0023              //jump to inner for

IL_0012:  ldarg.0                 //load arr
IL_0013:  ldloc.2                 //load j
IL_0014:  ldelem.i4               //load arr[j]
IL_0015:  ldarg.0                 //load arr
IL_0016:  ldloc.1                 //load least
IL_0017:  ldelem.i4               //load arr[least]
IL_0018:  bge IL_001f             //jmp to skip loop body if arr[j] > arr[least]

IL_001d:  ldloc.2
IL_001e:  stloc.1                 //store least = j
IL_001f:  ldloc.2
IL_0020:  ldc.i4.1
IL_0021:  add
IL_0022:  stloc.2                 //increment j
IL_0023:  ldloc.2                 //inner-for guard
IL_0024:  ldarg.0                 //load arr
IL_0025:  ldlen                   //replace with arr len
IL_0026:  conv.i4
IL_0027:  blt IL_0012             //jmp to <inner for body>, if len > arr[j]

IL_002c:  ldarg.0
IL_002d:  ldloc.0 
IL_002e:  ldelem.i4               //arr[i]
IL_002f:  stloc.3                 //saved to tmp
IL_0030:  ldarg.0
IL_0031:  ldloc.0 
IL_0032:  ldarg.0 
IL_0033:  ldloc.1 
IL_0034:  ldelem.i4               //arr[least]
IL_0035:  stelem.i4               //arr[i] =
IL_0036:  ldarg.0 
IL_0037:  ldloc.1 
IL_0038:  ldloc.3 
IL_0039:  stelem.i4               //arr[least] = tmp
IL_003a:  ldloc.0 
IL_003b:  ldc.i4.1 
IL_003c:  add 
IL_003d:  stloc.0                 //increment i
IL_003e:  ldloc.0                 //for guard, load i
IL_003f:  ldarg.0                 //push arr (array)
IL_0040:  ldlen                   //length of arr
IL_0041:  conv.i4
IL_0042:  blt IL_0007             //if enter, jmp to for body

IL_0047:  ret 
} // end of method Selsort::SelectionSort
\end{lstlisting}
\subsection*{9.1(ii) Selsort.jvmbytecode}
\begin{lstlisting}[numbers=left, firstnumber=0, title=Selsort.jvmbytecode]
  public static void SelectionSort(int[]); flags: ACC_PUBLIC, ACC_STATIC Code: stack=4, locals=4, args_size=1
0: iconst_0
1: istore_1                       //store 0
2: iload_1                        //for guard
3: aload_0                        //push arr
4: arraylength                    //push arr length
5: if_icmpge 57                   //jmp to halt if i > arr len
8: iload_1                        //push i
9: istore_2                       //save to least
10: iload_1                       //push i
11: iconst_1
12: iadd                          //i+1
13: istore_3                      //j=i+1
14: iload_3                       //inner-for guard
15: aload_0
16: arraylength
17: if_icmpge 37                  //if j > arrlen, jmp to <??>
20: aload_0                       //push arr
21: iload_3                       //and j
22: iaload                        //arr[j]
23: aload_0                       //arr
24: iload_2                       //least
25: iaload                        //arr[least]
26: if_icmpge 31                  //jmp to <??> if arr[j] > arr[least]
29: iload_3                       //push j       
30: istore_2                      //store to least
31: iinc 3, 1                     //j++
34: goto 14                       //jmp to inner-for guard
37: aload_0
38: iload_1
39: iaload                        //arr[i]
40: istore_3                      //tmp=arr[i]
41: aload_0
42: iload_1
43: aload_0
44: iload_2
45: iaload                        //arr[least]
46: iastore                       //arr[i] = arr[least]
47: aload_0
48: iload_2
49: iload_3
50: iastore                       //arr[least] = tmp
51: iinc 1, 1                     //i++
54: goto 2                        jmp to for guard
57: return
\end{lstlisting}

\subsection*{9.3 Queue with mistake}
The \emph{dummy} node is always in scope. The \emph{dummy.next} always refers to the
first element added the queue (or null, initially), regardless of whether or not
it has been popped. This holds for \emph{dummy.next.next} and the node added
secondly. (apply induction from here on out). This means that all added nodes
(again, regardless of whether they have been popped) are within scope as long as
the \emph{dummy} node is within scope, which is always.
A possible solution is to fix the \emph{.next} references when popping elements.
A slightly more elegant solution is to declare dummy in the constructor. That
way the dummy (and hence all popped nodes) will be out of scope, and hence
collectable by the garbage collector.

\end{document}
