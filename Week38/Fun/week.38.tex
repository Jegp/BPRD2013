\documentclass[a4paper]{article}

\usepackage{graphicx} % Required to insert images
\usepackage{natbib}
\usepackage{rotating}

\usepackage{amsmath}
\usepackage{tabularx}
\usepackage[utf8]{inputenc}

\usepackage[ligature,inference,reserved,shorthand]{semantic}

\usepackage{color}
\usepackage[usenames,dvipsnames,svgnames,table]{xcolor}


% Grammars
\def\grammar#1#2{%
  \vspace{1em}
  \begin{tabular}{r@{$\quad$}r@{$\quad$}l@{$\qquad$}l}
  $#1$ & $::=$ & & \emph{#2}\\
}
\def\endgrammar{\end{tabular}\vspace{1em}}
\def\prod#1#2{&$\mid$&$#1$&\emph{#2}}

% Meta
\newcommand{\RULE}[1]{\textsc{#1}}
\newcommand{\SUBST}[2]{\left[#1\right]#2}

% Syntax styles
\colorlet{term}{red}
\newcommand{\kwd}[1]{\ensuremath{\mathsf{\textcolor{term}{\underline{#1}}}}}
\newcommand{\expr}[1]{\ensuremath{\textcolor{term}{#1}}}
\newcommand{\type}[1]{\ensuremath{\textcolor{blue}{#1}}}
\newcommand{\context}[1]{\ensuremath{\textcolor{black}{#1}}}
\newcommand{\scheme}[1]{\ensuremath{\textcolor{ForestGreen}{#1}}}
\newcommand{\black}[1]{\textcolor{black}{#1}}

\newcommand{\const}[1]{\mathsf{#1}}

\newcommand{\typing}[3]{\ensuremath{#1\ \vdash\ #2\ :\ #3}}


% Object language syntax
\newcommand{\TRUE}{\ensuremath{\expr{\const{true}}}}
\newcommand{\FALSE}{\ensuremath{\expr{\const{false}}}}
\newcommand{\PLUS}[2]{#1\;\expr{+}\;#2}
\newcommand{\LT}[2]{#1\;\expr{<}\;#2}
\newcommand{\NUM}[1]{\expr{\const{#1}}}

\newcommand\var[1]{\ensuremath{\expr{\mathsf{#1}}}}
\newcommand{\app}[2]{#1\ #2}

\newcommand{\IF}[3]{\kwd{if}\ #1\ \kwd{then}\ #2\ \kwd{else}\ #3}

\newcommand{\LET}[3]{\kwd{let}\ #1\ \textcolor{term}{=}\ #2\ \kwd{in}\ #3\ \kwd{end}}
\newcommand{\LETF}[4]{\kwd{let}\ #1\ #2\ \textcolor{term}{=}\ #3\ \kwd{in}\ #4\ \kwd{end}}

% Object language types
\newcommand\INT{\type{\const{int}}}
\newcommand\BOOL{\type{\const{bool}}}
\newcommand{\fn}[2]{#1\;\type{\to}\;#2}

% Type schemes
\newcommand\FORALL[2]{\scheme{\forall #1\;.\;#2}}
\newcommand\ALPHAS[1][n]{\alpha_1,\ldots,\alpha_{#1}}

%Contexts
\newcommand{\EMPTY}{\context{\cdot}}
\newcommand{\WITH}[2]{\context{#1\left[#2\right]}}
\newcommand{\RHO}{\context{\rho}}
\newcommand{\LOOKUP}[2]{#1\left(#2\right)}


\title{A Tutorial on Polymorphic Type Derivations}
\author{David Christiansen}
\date{DRAFT of December 27, 2012}
\begin{document}

\title{Week 38}
\author{Sigurt Bladt Dinesen \\sidi{@}itu.dk \and Jens Egholm Pedersen \\jegp{@}itu.dk}
\maketitle
\section*{Exercise 6.1 (PLC)}
\subsection*{Is the result of the third one as expected?}
\begin{verbatim}
1 let add x = let f y = x+y in f end
2 in let addtwo = add 2
3    in let x = 77 in addtwo 5 end
4    end
5 end
\end{verbatim}
We would expect $add\ x$ to return a function $f\ of\ y$ that in turn returns x+y.
We would expect $addtwo$ to increment its (integer) input by 2, and the entire expression
to return 7.
The $let\ x = 77$ in line 3 means nothing. x from line 1 is not overwritten, only "shadowed".

\subsection*{Explain the result of the last one}
\begin{verbatim}
1 let add x = let f y = x+y in f end
2 in add 2 end
\end{verbatim}
The result is a function (described as a Closure in our object language) of
type $(int \to int)$ As in the previous expression, $add\ x$ returns a function
$f\ of\ y$ that returns $x+y$. The result of the entire expression is then the
function of an integer, that returns the sum of 2 and that integer.
In our object language, that is the closure
$$("f","y", Prim("+", Var "x", Var "y")$$
with an environment $$\rho [x \mapsto 2]$$
\textit{\small{(feel free to comment the notation.)}}

The environment also contains the closure that defines the $add\ x$ function,
though we will not need it to use the function returned by the expression.

\section*{Exercise 6.4 (PLC)}
\subsection*{(i)}
\begin{verbatim}
1 let f x = 1
2     in f f end
\end{verbatim}
Since we define $f\ f$ to be of type $t_x \to int$ and we define $f$ to be of type $t_x$ (since it is the argument to $f$), we have to define the argument for $f$ as a polymorphic type or risk infinity.
\\

\begin{displaymath}
  \scalebox{0.6}{
    \inference[\RULE{p8}]{
      \rho[x \mapsto t_x, f \mapsto t_x \to int] \vdash 1 : int &
      \inference[\RULE{p9}]{
        \inference[\RULE{p3}]{
          \rho (f) = \forall \alpha_1 ... \alpha_n.t_x
        }{
          \rho [f \mapsto \forall \alpha_1 ... \alpha_n.t_x \to int] \vdash f : t_x \to int
        } &
        \rho \mapsto f : t_x
      }{
        \rho[f \mapsto \forall \alpha_1 ... \alpha_n.t_x \to int] \vdash ff : t 
      } &
      \quad \alpha_1 ... \alpha_n not\ free\ in\ \rho
    }{
      \rho \vdash let\ f\ x = 1\ in\ f\ f\ end : t
    }
  }
\end{displaymath}

\subsection*{(ii)}
\begin{verbatim}
1 let f x = if x < 10 then 42 else f(x+1)
2     in f 20 end
\end{verbatim}

\end{document}
