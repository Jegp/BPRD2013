\documentclass{article}

\usepackage{graphicx} % Required to insert images
\usepackage{courier} % Required for the courier font

\begin{document}

\title{Week 38}
\author{Sigurt Bladt Dinesen \\sidi{@}itu.dk \and Jens Egholm Pedersen \\jegp{@}itu.dk}
\maketitle
\section*{Exercise 6.1 (PLC)}
\subsection*{Is the result of the third one as expected?}
\begin{verbatim}
1 let add x = let f y = x+y in f end
2 in let addtwo = add 2
3    in let x = 77 in addtwo 5 end
4    end
5 end
\end{verbatim}
We would expect $add\ x$ to return a function $f\ of\ y$ that in turn returns x+y.
We would expect $addtwo$ to increment its (integer) input by 2, and the entire expression
to return 7.
The $let\ x = 77$ in line 3 means nothing. x from line 1 is not overwritten, only "shadowed".

\subsection*{Explain the result of the last one}
\begin{verbatim}
1 let add x = let f y = x+y in f end
2 in add 2 end
\end{verbatim}
The result is a function (described as a Closure in our object language) of
type $(int \to int)$ As in the previous expression, $add\ x$ returns a function
$f\ of\ y$ that returns $x+y$. The result of the entire expression is then the
function of an integer, that returns the sum of 2 and that integer.
In our object language, that is the closure
$$("f","y", Prim("+", Var "x", Var "y")$$
with an environment $$\rho [x \mapsto 2]$$
\textit{\small{(feel free to comment the notation.)}}

The environment also contains the closure that defines the $add\ x$ function,
though we will not need it to use the function returned by the expression.
\end{document}
